% Main "look" of your document is determined by the documentclass.
% The default/standard one is "article".
% Other options include "paper" and "report".
% To get the VUB style, you can find instructions at https://gitlab.com/rubdos/texlive-vub 
\documentclass{article}

% Packages from the AMS, you will basically always need these.
% Amsmath gives acces to equation and align environments
% Amssymb loads some handy math symbols and fonts (like mathbb/mathfrak)
% Amsthm is used for defining theorem environments
\usepackage{amsmath, amssymb, amsthm, mathtools}
% Makes links/references in your document clickable.
\usepackage{hyperref}
% Babel is to ensure that table of contents, figures, references will get the correct name.
% \usepackage[dutch]{babel}

% Automatically inserts environment names into references.
%\usepackage[dutch]{cleveref}
\usepackage{cleveref}

% Easily change the enumeration type
% WARNING: doesn't work with beamer!
\usepackage[shortlabels]{enumitem}
\usepackage{graphicx}

% Theorem environments
\theoremstyle{plain}
\newtheorem{theorem}{Theorem}[section] % You can specify at what depth the counter is reset.
% The [theorem] indicates that it uses the same numbering counter as theorem.
\newtheorem{lemma}[theorem]{Lemma}
\newtheorem{prop}[theorem]{Proposition}
\newtheorem{corollary}[theorem]{Corollary}
\newtheorem*{theorem*}{Theorem}
\newtheorem*{lemma*}{Lemma}
\newtheorem*{prop*}{Proposition}
\newtheorem*{corollary*}{Corollary}

\theoremstyle{definition}
\newtheorem{definition}[theorem]{Definition}
\newtheorem{example}[theorem]{Example}
\newtheorem{exercise}[theorem]{Exercise}
\newtheorem*{definition*}{Definition}
\newtheorem*{example*}{Example}
\newtheorem*{exercise*}{Exercise}

\theoremstyle{remark}
\newtheorem{remark}[theorem]{Remark}
\newtheorem*{remark*}{Remark}

%% Macros
\newcommand{\R}{\mathbb{R}}
\newcommand{\N}{\mathbb{N}}
\newcommand{\Z}{\mathbb{Z}}
\newcommand{\Q}{\mathbb{Q}}
\newcommand{\C}{\mathbb{C}}

% Information for the title page
\title{A showcase of \LaTeX}
\author{Wannes Malfait, Andreas Lorrain}
% You can set the date to nothing, if you don't want it to appear.
% \date{}


\begin{document}

\maketitle
\newpage

\tableofcontents
\newpage

\section{Introduction}

This document contains a bunch of random \LaTeX{} code to give you an idea of what is
possible. Let us start with a theorem:
\begin{theorem}[Pythagoras] % You can give a name between [].
	\label{thm:pythagoras}
	In a right-angled triangle with short sides $a$ and $b$ and hypotenuse $c$,
	the following identity always holds:
	\begin{equation}\label{eq:pythagoras}
		a^2 + b^2 = c^2.
	\end{equation}
\end{theorem}
\begin{proof}
	This follows as a trivial consequence of \cite[Theorem 6.9]{GeissLeclercSchroer2008PFvar}.
\end{proof}

You can then refer to \Cref{thm:pythagoras} like so, and \cref{eq:pythagoras} like so.
You can even refer to things before you define them! Look at this beautiful image:
\cref{fig:meme}.
\begin{figure}
	\centering

	\includegraphics[width=0.7\columnwidth, trim={2mm 2mm 2mm 2mm},clip]{meme.png}

	\caption{A very accurate meme.}
	\label{fig:meme}
\end{figure}

\section{Math!}

Let us look at some examples of mathematical formulas.
\begin{itemize}
	\item Create a matrix with \texttt{pmatrix}. The ``p'' stands for parenthesis. You also have
	      \texttt{vmatrix} (for determinants) and \texttt{bmatrix} (vertical respectively
	      brackets).
	      \begin{equation*}
		      \begin{pmatrix}
			      5      & \alpha & \text{test} & 3 \\
			      0      & \ddots &             & 0 \\
			      \vdots &        &             & 0 \\
			      0      & 1      & 1           & 5
		      \end{pmatrix}
		      \begin{vmatrix}
			      5      & \alpha & \text{test} & 3 \\
			      0      & \ddots &             & 0 \\
			      \vdots &        &             & 0 \\
			      0      & 1      & 1           & 5
		      \end{vmatrix}
		      \begin{bmatrix}
			      5      & \alpha & \text{test} & 3 \\
			      0      & \ddots &             & 0 \\
			      \vdots &        &             & 0 \\
			      0      & 1      & 1           & 5
		      \end{bmatrix}
	      \end{equation*}
	\item The \texttt{align} environment allows you to align equations after each other:
	      \begin{align*}
		      |x_1+x_2+\dots +x_n|^2 & = |\sum_{i=1}^{n}x_i|^2      \\
		                             & \leq (\sum_{i=1}^{n}|x_i|)^2 \\
		                             & \leq \sum_{i=1}^n(x_i)^2
	      \end{align*}
	\item Writing out $\mathbb{R}$ can be tedious. Luckily we defined a ``macro'' to save us some
	      time. Wow $\R$, so easy!
	\item Sometimes your parenthesis look very small. Like here for example:
	      \begin{equation*}
		      x= (\frac{\sum_{i=0}^n x^i}{\sin \theta}).
	      \end{equation*}
	      You can use \textbackslash \texttt{left(} and \textbackslash \texttt{right)} for this:
	      \begin{equation*}
		      x= \left(\frac{\sum_{i=0}^n x^i}{\sin \theta}\right).
	      \end{equation*}
	\item Most mathematical symbols have a nice command:
	      \begin{equation*}
		      \alpha, \vartriangleleft, \subseteq, \int, \dots.
	      \end{equation*}
	      If you don't know the name of a command you can use \href{http://detexify.kirelabs.org/classify.html}{this website} to draw it, and it will let you know what the command is.
\end{itemize}

\begin{definition}
	A \emph{monad} is a monoid in the category of endofunctors\footnote{This is definition actually makes sense, even though it is used jokingly}.
\end{definition}
\begin{remark}
	I'm too lazy to explain what this means, but the relevant definitions can be found in any standard text book on category theory.
\end{remark}

% Follow the guidelines of your course for what style to use.
\bibliographystyle{apalike}
% Only papers you cite will appear.
% Otherwise you can use \nocite{PaperRef}.
\bibliography{refs.bib}

\end{document}
